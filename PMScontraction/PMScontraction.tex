% !TEX root = ../stellar-notes.tex

\section{The Jeans' criterion}

We'll mention briefly a classic piece of stability analysis, and that is of the collapse of a homogeneous, isotropic fluid. Our equations are conservation of mass and momentum, plus Poisson's equation for the gravitational potential:
\begin{eqnarray}
\partial_{t}\rho + \divr(\rho \vu) &=& 0\label{e.jeans-mass-cons}\\
\partial_{t}(\rho\vu) + \divr(\rho\vu\vu) &=& - \rho \grad\Phi - \grad P\label{e.jeans-momentum}\\
\nabla^{2} \Phi &=& 4\pi G \rho\label{e.jeans-poisson}.
\end{eqnarray}
Right away we run into a snag: if our system is homogenous and isotropic, then $\grad\Phi = 0$, since there is no preferred direction for the vector to point. In this case the left-hand side of Poisson's equation vanishes, which is inconsistent with our having a background density.  This, of course, is the central point to cosmology, and if we want to do this calculation correctly we need general relativity and an expanding background universe. 

Instead, we shall take an alternate route: following Jeans's lead, we simply assert that there is a background state of uniform density $\rho_{0}$. This not-quite-consistent approach still provides insight.
Forging ahead, we write the density, velocity, and potential as a background piece (subscript ``0'') plus a perturbation (subscript ``1''):
\begin{eqnarray*}
\rho(\vx,t) &=& \rho_{0} + \rho_{1}\exp(i\bvec{k}\vdot\vx - i\omega t)\\
\vu(\vx,t) &=& \vu_{1}\exp(i\bvec{k}\vdot\vx - i\omega t)\\
\Phi(\vx,t) &=& \Phi_{1}\exp(i\bvec{k}\vdot\vx - i\omega t)
\end{eqnarray*}
We take the perturbations to be adiabatic: $\dif P = c_{s}^{2}\dif\rho$. Substituting these into eqns.~(\ref{e.jeans-mass-cons})--(\ref{e.jeans-poisson}), dropping all terms that are higher than first order, dotting $\bvec{k}$ into the perturbed form of eq.~(\ref{e.jeans-momentum}) and using the other two equations to eliminate terms results in a dispersion relation,
\begin{equation}\label{e.jeans-dispersion}
\omega^{2} = c_{s}^{2}k^{2} - 4\pi G\rho_{0}.
\end{equation}
For sufficiently small $k$ (long wavelengths) $\omega^{2}<0$ and the perturbations will grow exponentially. Setting $\omega = 0$ defines the \newterm{Jeans' length},
\begin{equation}\label{e.Jeans-length}
\lambda_{\mathrm{J}} = c_{s}\sqrt{\frac{\pi}{G\rho_{0}}}.
\end{equation}
We recognize the right-hand side as just being $\sim c_{s} \tau_{\mathrm{dyn}}$: equation~(\ref{e.Jeans-length}) simply says that regions where the sound-crossing time are longer than the dynamical timescale are subject to collapse.  The mass contained in a box of size $\lambda_{\mathrm{J}}$ is $M_{\mathrm{J}} = c_{s}^{3}(\pi/G)^{3/2}/\rho^{1/2}$.  For conditions appropriate to the dense cores of molecular clouds---temperatures $\sim 10\nsp\K$, $\mathrm{H}_{2}$ densities $\sim 10^{3}\nsp\cm^{-3}$---the Jeans mass is $M_{\mathrm{J}} \sim 100\nsp\Msun$.

\section{The Hayashi Track}\label{s.Hayashi}

For a fully convective star, we can use the fact that the entropy per unit mass is the same at the photosphere as at the center to relate the surface temperature to the mass and radius of the star. Namely,
\begin{equation}\label{e.Teff-adiabat}
\Teff = T_{c}\left(\frac{P_{\mathrm{ph}}}{P_{c}}\right)^{2/5},
\end{equation}
with
\begin{eqnarray}
T_{c} &=& 0.5 \frac{GM\mu \mb}{kR} \label{e.fc-Tc}\\
P_{c} &=& 0.8\frac{GM^{2}}{R^{4}} \label{e.fc-Pc}
\end{eqnarray}
being the central density and pressure of a polytrope of index $n=3/2$, and $P_{\mathrm{ph}}$ being the root of the equation
\begin{equation}\label{e.pph}
 P_{\mathrm{ph}} \approx \frac{GM}{R^{2}}\frac{1}{\kappa_{0}(\mu\mb/k)^{r}P_{\mathrm{ph}}^{r}\Teff^{s-r}}. 
\end{equation}
(We could have used the solution for $P(\tau)$ from before, but this approximation is accurate enough to demonstrate our point.)

Now for some crazy fractions: insert equations~(\ref{e.fc-Tc}), (\ref{e.fc-Pc}), and (\ref{e.pph}) into equation~(\ref{e.Teff-adiabat}) and solve for $\Teff$ to find
\begin{equation}\label{e.Teff-MR}
\Teff^{5+3r+2s} = 0.55^{5(1+r)}\kappa_{0}^{-2} \left(\frac{G\mu\mb}{k}\right)^{5+3r}M^{3+r} R^{3r-1}.
\end{equation}
What does this say for Thomson scattering ($\kappa_{0} \approx 0.4$, $r=s=0$)? Inserting these values into eq.~(\ref{e.Teff-MR}) gives
\begin{equation}\label{e.Teff-th}
	\Teff \approx 250\nsp\K \left(\frac{M}{\Msun}\right)^{3/5}\left(\frac{R}{R_{\odot}}\right)^{-1/5},
\end{equation}
a ridiculous value. Let's try it with H$^{-}$ opacity ($\kappa_{0} \approx 2.5\ee{-31}$, $r=1/2$, $s=9$). In this case,
\begin{equation}\label{e.Teff-H-}
 \Teff \approx 2200\nsp\K \left(\frac{M}{\Msun}\right)^{1/7}\left(\frac{R}{\Rsun}\right)^{1/49}.
\end{equation}
Note the extremely weak dependence on $R$.  This is a consequence that $R$ is very sensitive to the entropy at the photosphere.  Writing $L = 4\pi R^{2}\ssb \Teff^{4}$, we can solve for $R$ and insert it into equation~(\ref{e.Teff-H-}) to get the effective temperature in terms of mass and luminosity,
\begin{equation}\label{e.Hayashi}
 \Teff \approx 2300\nsp\K \left(\frac{M}{\Msun}\right)^{7/51}\left(\frac{L}{L_{\odot}}\right)^{1/102}.
\end{equation}
This is a crude estimate, so don't take these numbers too seriously. What this exercise illustrates, however, is that the effective temperature for fully convective, low-mass stars is essentially independent of luminosity.  On an HR diagram, these stars follow a vertical track, known as the \emph{Hayashi track}, as they contract to the main sequence.

\section{Formation of a radiative core}

For low-mass stars, the opacity in the core has a Kramers'-like form, $\kappa \propto \rho T^{-7/2}$. From our scalings, we see that $\kappa \propto M^{-5/2} R^{1/2}$.  As a result, as the star contracts, the central opacity decreases.  If it decreases enough, then photons can carry the flux and a radiative core will develop.

To find out when a radiative core forms, let's start with our equation for the flux,
\[ F(r) = -\frac{1}{3}\frac{c}{\rho\kappa}\frac{\dif aT^{4}}{\dif r}, \]
and use hydrostatic balance to rewrite this as
\[ L(r) = 4\pi r^{2}F(r) = \frac{16\pi}{3}  \frac{acT^{4}}{\kappa} \frac{Gm(r)}{P} \frac{\dif \ln T}{\dif \ln P}. \]
Now imagine we consider a tiny amount of matter $\delta m$ about the center of the star.  The luminosity coming out of this sphere is $\delta L(r)$,  and since in a convectively stable atmosphere $\dif\ln T/\dif\ln P < (\Gamma_{2}-1)/\Gamma_{2}$, the maximum amount of energy that can be generated in the sphere $\delta m$ and transported away in the absence of convection is
\begin{equation}\label{e.max_deltaL}
  \frac{\delta L(r)}{\delta m} < \frac{16\pi}{3} \frac{Gac}{\kappa} \frac{T_{c}^{4}}{P_{c}}  \frac{\Gamma_{2}-1}{\Gamma_{2}}. 
\end{equation}
Flashback! Do you remember doing problem \ref{p.lagrange-heat} of chapter \ref{ch.stellar-structure-eqn}? This gave us an expression, eq.~(\ref{e.lagrange-heat-alt2}), for $\partial L/\partial m$, which we can equate with the LHS of equation~(\ref{e.max_deltaL}):
\[
 q - \frac{P_{c}}{\rho_{c}(\Gamma_{3}-1)}\frac{\Dif }{\Dif t}\ln\left(\frac{P_{c}}{\rho_{c}^{\Gamma_{1}}}\right) <
 	\frac{16\pi}{3} \frac{Gac}{\kappa} \frac{T_{c}^{4}}{P_{c}}  \frac{\Gamma_{2}-1}{\Gamma_{2}}.
\]
Let's do the time warp again: in problem \ref{p.radius-entropy} of chapter \ref{ch.polytropes}, you showed that $P/\rho^{\Gamma_{1}} \propto R$ for an ideal gas (i.e., the polytropic constant $K$); making this substitution in the derivative gives the condition for the formation of a radiative core,
\begin{equation}\label{e.formation-radiative-core}
q - \frac{\NA\kB T_{c}}{\mu(\Gamma_{3}-1)} \frac{\Dif \ln R}{\Dif t} < \frac{16\pi}{3} \frac{Gac}{\kappa} \frac{T_{c}^{4}}{P_{c}}  \frac{\Gamma_{2}-1}{\Gamma_{2}}.
\end{equation}
Stars with $M \gtrsim 0.3 \Msun$ form a radiative core while contracting to the main-sequence; in contrast, lower-mass stars remain fully convective throughout their main-sequence lives.

\begin{exercisebox}[Contraction of a low-mass pre-main-sequence star]
We showed that low-mass pre-main-sequence stars (including brown dwarfs) are fully convective and have nearly constant effective temperatures.  Use these facts to model their pre-main sequence contraction.  Assume $T_{\mathrm{eff}} = \mathrm{const.}$ so that the luminosity is $L = 4\pi R^{2} \sigma_{\mathrm{SB}} T_{\mathrm{eff}}^{4}$.

\begin{enumerate}
\item Use the appropriate polytropic relation for the energy of the protostar and assume that the luminosity is entirely powered by contraction, i.e., the star is not yet approaching the main-sequence. Derive an equation for $R(t)$. What is the characteristic timescale for a low-mass star to contract? Scale your answer to $T_{\mathrm{eff}} = 3000\nsp\K$ and $M = 0.1\nsp\Msun$ (i.e., get an analytical solution in terms of the variables $\tilde{T} = [\Teff/3000\nsp\K]$ and $\tilde{M} = [M/0.1\nsp\Msun]$).

\item Compare your findings with more elaborate calculations.  You will find a review in \href{http://arxiv.org/abs/astro-ph/0006383}{``Theory of Low-Mass Stars and Substellar Objects,''} G. Chabrier and I. Baraffe, Ann.\ Rev.\ Astron.\ Astrophys.\ \textbf{38:} 337 (2000). 
%\item Using appropriate expressions for the central density and temperature, calculate the time required for a star just below the hydrogen burning limit (about $0.07\nsp\Msun$) to contract to a radius such that $\eF(\rho_{c}) = \kB T_{c}$, where $\eF$ is the Fermi energy, and compare with the evolutionary tracks in Chabrier \& Baraffe.
\end{enumerate}
\end{exercisebox}
