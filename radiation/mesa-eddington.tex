% !TEX root = ../stellar-notes.tex

\DefineShortVerb{\|}

\section*{\raisebox{-0.015ex}{\includegraphics[height=1.4ex]{mesa_logo2}} Radiation pressure and the Eddington luminosity for massive stars}

Construct zero-age main-sequence (ZAMS) stars of masses $\val{1.0}{\Msun}$, $\val{3.0}{\Msun}$, $\val{10.0}{\Msun}$, and $\val{30.0}{\Msun}$. You will find the template for the project in  the folder |radiation/beta-eddington|. 

\begin{enumerate}
\item For each star, plot $\beta\equiv \Pgas/P$ as a function of Lagrangian mass coordinate $m$. Is $\beta$ roughly constant, i.e., independent of $m$? For each ZAMS model, assign a ``typical'' value of $\beta$ and plot this $\beta$ as a function of the total stellar mass $M$.  How well does $\beta(M)$ agree with what you derived in exercise~\ref{p.radiation-beta}?

\item For each star, plot $\Lrad/\Ledd$ as a function of $m$.
\end{enumerate}
The template project files are set up to load a file `|inlist_radn_vars|', which you will write.  The file should contain a customized version of `|Profile_Panels1|' that displays the $\beta$ and $\Lrad/\Ledd$. Note that the values of $\beta$ and $\Lrad/\Ledd$ are not output by default, so you will need to add them to the list of columns in the `profile' data files, as we did in previous exercises.

Finally, you may notice that things get interesting near the surface of the star, especially for the more massive stars.  Change the independent variable from `|mass|' to `|logxq|' and redo the plots. (You may need to adjust the minimum value of the x-axis and reverse the direction of the x-axis.) Comment on the results.

\UndefineShortVerb{\|}
